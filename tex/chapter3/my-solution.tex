\section{Overview}

This chapter details my solution to the problem.
The traditional software engineering task-based organization is used.
Section 1 describes the requirements of the solution, focusing on features.
Section 2 describes the architecture of the solution.
Section 3 describes the implementation of the solution.
Hardware and software topics are covered in section 3, from physical installation to the graphical user interface.
Section 4 describes the testing regime undertaken during the development of this project.

\subfile{chapter3/requirements}

\section{Metrics}

\subsection{Monetary Cost}

This project relies entirely upon free software.
The hardware reference platform's bill of materials is the only monetary cost associated with the project.
The intent of this project is to undercut the cost of existing solutions that are available to retail customers.

\subsection{Boot Time}

The time for a device to start up from a powered-off state will be measured and compared to other comparable systems such as mobile phones and personal computers.
The goal is to achieve a quick boot time that is acceptable to automotive end-users and effort will be taken to minimize this value.

\subsection{Filesystem Size}

While storage is inexpensive, this project doesn't intend to store a large amount of data.
Furthermore, it intends to include as little as possible code to maximize performance and maintainability.
Efforts will be made to minimize the filesystem size and occupy much less storage space than existing mobile operating systems.

\subfile{chapter3/hardware}

\section{Software Architecture}

\subsection{Platform}

The project was designed with simplicity, performance, and modularity in mind.
Its target hardware is cheap single-board computers.
These motivations influenced the choice of a development platform.

\subsection{Organization}

The project is designed into several service components, each with a well-defined role and programming interface.
This type of organization allows for independent development of components, a clear separation of concerns, and autonomous behavior of each component.
If the tools are also given their own process space in a modern multi-process operating system such as Linux, there is also the potential for increased security and robustness against failures in one component.
In fact, the application code of this project was designed as a set of interfaces for the D-Bus interprocess messaging system \cite{dbus}.

\subfile{chapter3/implementation}
